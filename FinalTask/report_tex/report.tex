\documentclass[12pt, a4paper]{jarticle}
\setlength{\topmargin}{0cm}
\setlength{\oddsidemargin}{5mm}
\setlength{\evensidemargin}{5mm}
\setlength{\textheight}{23cm}
\setlength{\textwidth}{38zw}
\setlength{\parindent}{1zw}

\usepackage[dvipdfmx]{graphicx,color}
\usepackage{amsthm}
\usepackage{amsmath,amssymb} 
\usepackage{here}
\usepackage{alltt}
\usepackage{enumerate}

\title{各国のモバイル端末のOSのシェアとGDP}
\author{226x105x 大森嶺}
\date{2022年6月12日}
\begin{document}
\maketitle

\section{導入}
現在,日常の様々な部分で多くの人がスマートフォンのアプリや,サービスを利用している.そのため,アプリケーションや,ソフトを開発する上で,OSのシェアについて把握,理解できていることは重要である.
また,モバイル端末のOSのシェア率について,世界の多くの国ではAndroidの占める割合が高いが,日本ではiOSの占める割合が高いと言われている.

そこで,日本,他の国についてもOSのシェア,その要因について,各国のOSのシェア率,人口,一人当たりの GDP を可視化することで,その関係や特徴を調べる.

\section{手法}
あああああああああああ

\section{結果}
あああああああああああ

\section{考察}
あああああああああああ

\section{結論}
あああああああああああ

\section{参照}
あああああああああああああああああ

\end{document}